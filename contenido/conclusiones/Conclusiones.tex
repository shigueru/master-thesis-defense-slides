\begin{frame}{Conclusiones}
    \begin{itemize}
    	\item Se han realizado con \'exito c\'alculos de estructura electr\'onica para el $BiFeO_{3}$ y el $YCrO_{3}$.
        \item Para el $BiFeO_{3}$ se observaron gaps de energ\'ia de $1.4$ eV y 
        $1.8$ eV para los arreglos antiferromagn\'eticos tipo A y G 
        respectivamente.
         \item Los orbitales \textbf{d} del hierro poseen la mayor densidad 
         cerca del nivel de fermi en la banda de conducci\'on.
        \item Para el $YCrO_{3}$ se observaron gaps de energ\'ia de $1.3$ eV, 
        $1.32$ eV y $1.6$ eV para los arreglos antiferromagn\'eticos tipo A, C 
        y G respectivamente.
        \item Los orbitales \textbf{d} del cromo poseen la mayor densidad cerca 
        del nivel de fermi en la banda de conducci\'on y en la banda de 
        valencia.
    \end{itemize}
\end{frame}

\begin{frame}
    Esta tesis a dado lugar a los siguientes trabajos:
    \begin{itemize}
        \item Un poster en XVII Encuentro de F\'isica.
        \item Un poster en XXVII Simposio Peruano de F\'isica.
        \item Un art\'iculo en la revista REVCIUNI.
    \end{itemize}
\end{frame}

\begin{frame}
    
\end{frame}